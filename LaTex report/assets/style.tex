% Classe del documento
\documentclass{article}

\usepackage[a4paper,headsep=20pt, top=50pt, bottom=50pt, left=80pt, right=80pt, footskip=20pt]{geometry}

% Pacchetto per generare testo segnaposto (Lorem Ipsum)
\usepackage{lipsum}

% Pacchetto per evidenziare codice sorgente
\usepackage{listings}

% Pacchetti per la matematica avanzata (font e simboli)
\usepackage{amsfonts}  % Font matematici aggiuntivi
\usepackage{amsmath}   % Ambienti ed estensioni per la matematica

% Pacchetto per personalizzare gli elenchi numerati e puntati
\usepackage{enumitem}

% Pacchetto per il grassetto nei simboli matematici
\usepackage{bm}

% Pacchetto per operazioni matematiche su lunghezze e dimensioni
\usepackage{calc}

% Pacchetto per gestire immagini e grafica
\usepackage{graphicx}

% Pacchetto per ridimensionare simboli matematici
\usepackage{relsize}

\usepackage{fancyhdr}			%used for head and foot settings

% Pacchetti per gestire colori e creare box colorati
\usepackage{xcolor}     % Definizione e uso di colori personalizzati
\usepackage{tcolorbox}  % Creazione di box colorati per evidenziare contenuti

% Pacchetto per evidenziare il testo
\usepackage{soul} 
\sethlcolor{red}


\fancypagestyle{plain}{
	\fancyhead{}
	\fancyfoot{}
	\fancyhead[L]{Vici Francesco}
	\fancyhead[R]{Laboratorio di Algoritmi e Strutture Dati}
	\fancyfoot[C]{\thepage}
}

% Definizione del percorso alla cartella immagini
\graphicspath{{assets/images/}}



\newenvironment{CENTER}
{
	\par
	\centering
	\setlength{\parskip}{10pt} % Imposta la spaziatura tra paragrafi
	\noindent
	%\Large  % Testo di dimensione grande
}
{
	\par
	\vspace{8pt} % Aggiunge spazio alla fine dell'ambiente
	\normalsize  % Ripristina la dimensione normale del testo
}

% Ambiente personalizzato per una lista descrittiva senza spaziature aggiuntive
\newenvironment{DESCRIPTION}
{
	\begin{description}
}
{
	\end{description}
}

% Variante di DESC con parametro per la gestione del margine sinistro
\newenvironment{DESC}[1]
{
	\setlist[description,1]{leftmargin=2em + (2em * #1), labelindent=0em + (2em * #1)}
	\begin{description}
}
{
	\end{description}
}

% Ambiente per una coppia di minipage con testo allineato in alto a sinistra
\newenvironment{MINIBOX}
{
	\begin{minipage}[t]{0.45\textwidth}
}
{
	\end{minipage}
}

\newenvironment{MINIBOX-RIGHT}
{
	\begin{minipage}[t]{0.45\textwidth}
		\begin{flushright}
}
{
		\end{flushright}
	\end{minipage}
}

% Definizione dell'ambiente con bordo e sfondo personalizzati
\newtcolorbox{GRAYBOX}{
	colframe=gray,    % Bordo grigio scuro
	colback=gray!10, % Sfondo grigio chiaro
	boxrule=1pt,      % Spessore del bordo
	arc=5pt,          % Arrotondamento angoli
	left=5pt, right=5pt, top=5pt, bottom=5pt % Margini interni
}

% Comando per aggiungere un watermark personalizzato al documento
\newcommand{\WATERMARK}[1]
{%
	\usepackage{draftwatermark}  % Carica il pacchetto per il watermark
	\SetWatermarkText{#1}        % Imposta il testo del watermark
	\SetWatermarkScale{4}        % Definisce la scala del watermark
	\SetWatermarkAngle{60}       % Imposta l'angolo di inclinazione del watermark
}

% Comando per inserire un'immagine centrata con spaziatura sopra e sotto
\newcommand{\IMG}[1]
{
	\vspace{1.5em}  % Aggiunge spazio sopra l'immagine
	\centering\includegraphics[width=1\columnwidth]{#1} % Inserisce l'immagine a tutta larghezza
	\vspace{1.5em}  % Aggiunge spazio sotto l'immagine
}

% Comando personalizzato per horizontal rule
\newcommand{\HR} 
{
	\vspace{1.5em}  % Aggiunge spazio sopra l'immagine
	\hrule
	\vspace{1.5em}  % Aggiunge spazio sotto l'immagine
}

% Titolo principale
\newcommand{\TITLE}[1]{\textsc{\LARGE #1}} 

% Sottotitolo
\newcommand{\SUBTITLE}[1]{\textsc{\Large #1}} 



\pagestyle{plain}

