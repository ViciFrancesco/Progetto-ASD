\usepackage[a4paper,headsep=20pt, top=50pt, bottom=50pt, left=80pt, right=80pt, footskip=20pt]{geometry}

% Pacchetti utili
\usepackage{lipsum}         % Testo segnaposto (Lorem Ipsum)
\usepackage{listings}       % Evidenziazione codice sorgente
\usepackage{amsfonts}       % Font matematici aggiuntivi
\usepackage{amsmath}        % Ambienti ed estensioni per la matematica
\usepackage{enumitem}       % Personalizzazione elenchi numerati e puntati
\usepackage{bm}             % Grassetto nei simboli matematici
\usepackage{calc}           % Operazioni matematiche su lunghezze e dimensioni
\usepackage{graphicx}       % Gestione immagini e grafica
\usepackage{relsize}        % Ridimensionamento simboli matematici
\usepackage{fancyhdr}       % Personalizzazione intestazioni e piedi di pagina
\usepackage{xcolor}         % Definizione e uso di colori personalizzati
\usepackage{tcolorbox}      % Creazione di box colorati per evidenziare contenuti
\usepackage{soul}           % Evidenziazione testo
\sethlcolor{yellow}            % Imposta il colore dell'evidenziazione

% Impostazioni dell'intestazione e del piè di pagina
\fancypagestyle{plain}{
	\fancyhead{}
	\fancyfoot{}
	\fancyhead[L]{Vici Francesco}
	\fancyhead[R]{Laboratorio di Algoritmi e Strutture Dati}
	\fancyfoot[C]{Pagina \thepage}
}

% Percorso cartella immagini
\graphicspath{{assets/images/}}

% Ambiente per centratura con spaziatura personalizzata
\newenvironment{CENTER}{
	\par\centering\setlength{\parskip}{10pt}\noindent
}{
	\par\vspace{8pt}\normalsize
}

% Ambiente per descrizioni senza spaziature aggiuntive
\newenvironment{DESCRIPTION}{
	\begin{description}
	}{
	\end{description}
}

% Variante di DESCRIPTION con margine personalizzabile
\newenvironment{DESC}[1]{
	\setlist[description,1]{leftmargin=2em + (2em * #1), labelindent=0em + (2em * #1)}
	\begin{description}
	}{
	\end{description}
}

% Minipage per layout con testo allineato a sinistra o destra
\newenvironment{MINIBOX}{
	\begin{minipage}[t]{0.45\textwidth}
	}{
	\end{minipage}
}

\newenvironment{MINIBOX-RIGHT}{
	\begin{minipage}[t]{0.45\textwidth}
		\begin{flushright}
		}{
		\end{flushright}
	\end{minipage}
}

% Ambiente per tabelle con caption personalizzabile
\newcommand{\tablecaption}{}
\newenvironment{TABLE}[2]{
	\begin{table}[ht]
		\centering
		\renewcommand{\tablecaption}{#2}  % Salva il secondo parametro
		\begin{tabular}{#1}
			\hline
		}{
			\hline
		\end{tabular}
		\vspace{5pt}
		\caption{\tablecaption}  % Usa il valore salvato
		\label{tab:tabella \thetable}
	\end{table}
}

\newenvironment{SIMPLETABLE}[1]{
	\begin{table}[h]
		\centering
		\begin{tabular}{#1}
		}{
		\end{tabular}
		\label{tab:tabellasemplice}
	\end{table}
}

% Box con bordo e sfondo personalizzati
\newtcolorbox{GRAYBOX}{
	colframe=gray, colback=gray!10,
	boxrule=1pt, arc=5pt,
	left=5pt, right=5pt, top=5pt, bottom=5pt
}
\newtcolorbox{ORANGEBOX}{
	colframe=orange, colback=orange!8,
	boxrule=1pt, arc=5pt,
	left=5pt, right=5pt, top=5pt, bottom=5pt
}

% Comando per watermark personalizzato
\newcommand{\WATERMARK}[1]{
	\usepackage{draftwatermark}
	\SetWatermarkText{#1}
	\SetWatermarkScale{4}
	\SetWatermarkAngle{60}
}

% Inserimento immagine centrata
\newcommand{\IMG}[1]{
	\vspace{1.5em}
	\centering\includegraphics[width=1\columnwidth]{#1}
	\vspace{1.5em}
}

% Linea orizzontale con spaziatura personalizzata
\newcommand{\HR}{
	\vspace{1.5em}
	\hrule
	\vspace{1.5em}
}

% Comandi per titoli e sottotitoli
\newcommand{\TITLE}[1]{\textsc{\LARGE #1}}
\newcommand{\SUBTITLE}[1]{\textsc{\Large #1}}

% Imposta lo stile di pagina
\pagestyle{plain}