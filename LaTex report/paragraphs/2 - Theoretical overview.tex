\section{Panoramica Teorica}

	\subsection{Introduzione al problema della Longest Common Subsequence (LCS)}
	L'algoritmo LCS, che sta per Longest Common Subsequence, consiste nel trovare la sottosequenza più lunga partendo da due stringhe qualsiasi, X e Y, date in input. Definisco inoltre come sottosequenza x di X una stringa che ha 0 o più caratteri in meno rispetto ad X. Inoltre è importante comprendere che l'ordine dei caratteri delle sequenze di partenza è fondamentale, altrimenti si parlerebbe di sottoinsiemi e non di sottosequenze. 
	
	\subsection{Diverse implementazioni della LCS}
	Esistono numerosi metodi d'implementazione dell'algoritmo LCS, quelli utilizzati in questo esperimento saranno i seguenti:
	\begin{DESC}{1}
		\item[\textbf{versione con algoritmo 'Brute Force'}] $\rightarrow$ utilizza il processo di forza bruta per ottenere tutte le sottosequenze (di qualsiasi dimensione) di una delle due stringhe date in input, poi ne verifica la presenza nell'altra stringa in input.
		\item[\textbf{versione ricorsiva}] $\rightarrow$ scompone il problema di partenza in coppie di sottoproblemi più semplici, fino ad arrivare al caso base ovvero quando una delle due stringhe date in input ha dimensione 0.
		\item[\textbf{versione con memoization}] $\rightarrow$ per la risoluzione del problema utilizza \hl{lo stesso meccanismo della versione ricorsiva}, ma durante la risoluzione memorizza in una matrice i le sottosequenze comuni già trovate e le utilizza per evitare di effettuare lo stesso processo più volte. Infatti ad ogni iterazione se per gli input che riceve ha già individuato una soluzione la riporta senza effettuare nuovamente il processo.
		\item[\textbf{versione bottom-up}] $\rightarrow$ \hl{risolve i problemi a partire dalla dimensione minima e aumentandola ad ogni esecuzione. La risoluzione avviene 'al contrario' rispetto alle altre versioni.}
		
	\end{DESC}
	\subsection{Costi delle diverse implementazioni}