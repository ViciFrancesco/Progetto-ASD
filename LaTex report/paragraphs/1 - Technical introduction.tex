\section{Introduzione tecnica}

	\subsection{Esperimento assegnato}
		L'esperimento assegnato consiste nel confrontare vari modi per calcolare la LCS (Longest Common Subsequence) tra due stringhe:
		\begin{itemize}
			\item versione che utilizza l'algoritmo "forza-bruta"
			\item versione ricorsiva
			\item versione ricorsiva con memoization
			\item versione bottom-up
		\end{itemize}
	
	\subsection{Specifiche tecniche del dispositivo usato}
		Poiché l'esperimento si incentrerà sui tempi di esecuzione degli algoritmi appena elencati, è indispensabile la descrizione delle specifiche hardware del computer usato per effettuare i test. Di seguito le specifiche:
			\begin{DESC}{1}
				\item[\textbf{CPU}] $\rightarrow$ \hl{**nome componente**},
				\item[\textbf{RAM}] $\rightarrow$ \hl{**nome componente**},
				\item[\textbf{SSD}] $\rightarrow$ \hl{**nome componente**},
				\item[\textbf{Hard Disk}] $\rightarrow$ \hl{**nome componente**}.
			\end{DESC}
		Mentre gli strumenti software utilizzati nella realizzazione del progetto e della relazione sono:
			\begin{DESC}{1}
				\item [\textbf{Linguaggio di programmazione}] $\rightarrow$ Python 3.13.x,
				\item[\textbf{IDE}] $\rightarrow$ Visual Studio Code (latest release),
				\item[\textbf{Editor LaTex}] $\rightarrow$ TeXstudio (latest release),
				\item[\textbf{Compilatore LaTex}] $\rightarrow$ TeX Live 2023. 
			\end{DESC}
			
	\subsection{Come affrontare l'esperimento}
		La relazione dell'esperimento è costituita da 4 parti fondamentali:
		\begin{itemize}
			\item \textbf{Breve spiegazione teorica del problema}: oltre al testo dell'esercizio viene fornito un sommario delle principali caratteristiche teoriche del problema in esame, partendo dal materiale fornito durante il corso di Algoritmi e Strutture Dati ed ampliandolo con materiale esterno.
			\item \textbf{Documentazione del codice}: sono riportati i frammenti più importanti del codice Python insieme con uno schema UML di tutte le classi e una breve spiegazione delle struttura del progetto.
			\item \textbf{Descrizione dell'esperimento}: viene ripercorso lo svolgimento dell'esperimento con l'esposizione dei risultati ottenuti.
			\item \textbf{Analisi dei risultati e conclusioni}: i dati ottenuti nel corso dell'esperimento vengono messi in relazione con i risultati teorici attesi ed esposizione di una tesi conclusiva.
		\end{itemize}
		